\documentclass[report.tex]{subfiles}


\lstset{frame=tb,
  language=MATLAB,
  aboveskip=5mm,
  belowskip=5mm,
  showstringspaces=false,
  columns=flexible,
  basicstyle={\small\ttfamily},
  numbers=none,
  numberstyle=\tiny\color{gray},
  keywordstyle=\color{blue},
  commentstyle=\color{dkgreen},
  stringstyle=\color{mauve},
  breaklines=true,
  breakatwhitespace=true,
  tabsize=8
}

\begin{document}

        We calculated the average energy in a given window. We used the rolling
        technique so that the chances of missing out any data is less. We used
        the \emph{Parseval's Theorem} to calculate the average energy over the
        window.

        According to \emph{Parseval's Theorem}, if $x[k]$ and $X[f]$ are the
        pair of discrete time Fourier sequences, where $x[k]$ is the discrete
        time sequence and $X[f]$ is its corresponding DFT, the energy of the
        aperiodic sequence of length can be expressed in terms of its N-point
        DFT as follows:
        $$
        E_x = \frac{1}{N} \sum_{f = 0}^{N - 1} \bigg | X[f] \bigg |^2
        $$

        \subsection{Implementation in MATLAB}
        \begin{lstlisting}
        % Apply Parseval's Theorem to calculate the
        % instantaeneous energy(power) of the input signal

        function [pars] = Energy(in, win_sz, roll_factor)
        % Gives the instantaneous energy (power) of the
        % signal according to the win_sz and the roll_factor

                roll_sz = floor(win_sz * roll_factor);
                
                n = length(in);
                pars = zeros(size(in));
                
                % Make 'n' a multiple of window size
                n = win_sz * floor(n / win_sz);
                
                for i = 1:roll_sz:(n - win_sz)
                        x = in(i : i+win_sz);
                        
                        y = fft(x);     % Apply Fast Fourier Transform
                        amp = abs(y);   % Get coefficients after 
                                        % transformation

                        % Calculate Power using Parseval's Theorem
                        % E = sum |X(f)|^2
                        % P = (1 / N) * E
                        pars(i) = amp' * amp / win_sz;
                        
                end

        end
        \end{lstlisting}
        
\end{document}