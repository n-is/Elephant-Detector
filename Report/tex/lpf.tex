\documentclass[report.tex]{subfiles}


\lstset{frame=tb,
  language=MATLAB,
  aboveskip=5mm,
  belowskip=5mm,
  showstringspaces=false,
  columns=flexible,
  basicstyle={\small\ttfamily},
  numbers=none,
  numberstyle=\tiny\color{gray},
  keywordstyle=\color{blue},
  commentstyle=\color{dkgreen},
  stringstyle=\color{mauve},
  breaklines=true,
  breakatwhitespace=true,
  tabsize=8
}

\begin{document}

        The transfer function of the low pass filter is :
        
        $$
        H(s) = \frac{1}{1 + sRC}
        $$
        Discretizing using Tustin's approximation,
        $ s \leftarrow \frac{2}{T_s} \frac{1 - z^{-1}}{1 + z^{-1}} $

        $$
        H(z) = \frac{T_s (1 + z^{-1})}{T_s (1 + z^{-1}) + 2 (1 - z^{-1}) RC}
        $$
        If $x(t)$ be the input to the filter and $y(t)$ be the output produced
        by the filter, the above filter in time domain is :
        $$
        y(t) = \frac{1}{T_s + 2RC} \Bigg [T_s \Bigg (x(t) + x(t-1)\Bigg) +
        (2RC - T_s) y(t-1) \Bigg ]
        $$
        We can easily implement the above filter's form in code. Here $T_s$ is
        the sampling period.

        \subsection{Implementation in MATLAB}
        \begin{lstlisting}
        % Transfer Function of RC circuit
        % H = 1 / (1 + sRC)

        function [out] = tust_lpf(in, time, fc)
        % Apply Low Pass Filter using Tustin's Approximation

                out = zeros(size(in));
                out(1) = in(1);
                
                RC = 1 / (2*pi*fc);
                
                for i = 2:length(in)
                        dt = time(i) - time(i-1);
                        a = 1 / (dt + 2*RC);
                        b = (2*RC - dt);
                        out(i) = a * (dt * (in(i) + in(i-1)) ...
                                + b * out(i-1));
                end

        end
        \end{lstlisting}
        
\end{document}